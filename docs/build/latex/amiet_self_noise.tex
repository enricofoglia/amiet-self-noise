%% Generated by Sphinx.
\def\sphinxdocclass{report}
\documentclass[letterpaper,10pt,english]{sphinxmanual}
\ifdefined\pdfpxdimen
   \let\sphinxpxdimen\pdfpxdimen\else\newdimen\sphinxpxdimen
\fi \sphinxpxdimen=.75bp\relax
\ifdefined\pdfimageresolution
    \pdfimageresolution= \numexpr \dimexpr1in\relax/\sphinxpxdimen\relax
\fi
%% let collapsible pdf bookmarks panel have high depth per default
\PassOptionsToPackage{bookmarksdepth=5}{hyperref}

\PassOptionsToPackage{booktabs}{sphinx}
\PassOptionsToPackage{colorrows}{sphinx}

\PassOptionsToPackage{warn}{textcomp}
\usepackage[utf8]{inputenc}
\ifdefined\DeclareUnicodeCharacter
% support both utf8 and utf8x syntaxes
  \ifdefined\DeclareUnicodeCharacterAsOptional
    \def\sphinxDUC#1{\DeclareUnicodeCharacter{"#1}}
  \else
    \let\sphinxDUC\DeclareUnicodeCharacter
  \fi
  \sphinxDUC{00A0}{\nobreakspace}
  \sphinxDUC{2500}{\sphinxunichar{2500}}
  \sphinxDUC{2502}{\sphinxunichar{2502}}
  \sphinxDUC{2514}{\sphinxunichar{2514}}
  \sphinxDUC{251C}{\sphinxunichar{251C}}
  \sphinxDUC{2572}{\textbackslash}
\fi
\usepackage{cmap}
\usepackage[T1]{fontenc}
\usepackage{amsmath,amssymb,amstext}
\usepackage{babel}



\usepackage{tgtermes}
\usepackage{tgheros}
\renewcommand{\ttdefault}{txtt}



\usepackage[Bjarne]{fncychap}
\usepackage{sphinx}

\fvset{fontsize=auto}
\usepackage{geometry}


% Include hyperref last.
\usepackage{hyperref}
% Fix anchor placement for figures with captions.
\usepackage{hypcap}% it must be loaded after hyperref.
% Set up styles of URL: it should be placed after hyperref.
\urlstyle{same}

\addto\captionsenglish{\renewcommand{\contentsname}{Contents:}}

\usepackage{sphinxmessages}
\setcounter{tocdepth}{1}



\title{amiet\_self\_noise}
\date{Aug 15, 2025}
\release{0.1.0}
\author{Enrico Foglia, Odyssée Cadoret}
\newcommand{\sphinxlogo}{\vbox{}}
\renewcommand{\releasename}{Release}
\makeindex
\begin{document}

\ifdefined\shorthandoff
  \ifnum\catcode`\=\string=\active\shorthandoff{=}\fi
  \ifnum\catcode`\"=\active\shorthandoff{"}\fi
\fi

\pagestyle{empty}
\sphinxmaketitle
\pagestyle{plain}
\sphinxtableofcontents
\pagestyle{normal}
\phantomsection\label{\detokenize{index::doc}}


\sphinxAtStartPar
\sphinxcode{\sphinxupquote{amiet\sphinxhyphen{}self\sphinxhyphen{}noise}} is a library built to help researchers and engineers to predict airfoil self\sphinxhyphen{}noise from experimental or simulation data using the Amiet model.

\begin{sphinxadmonition}{note}{Note:}
\sphinxAtStartPar
This project is being developed as a final project for the course GMC729 (Aeroacoustics) at the University of Sherbrooke, under the supervision of Prof. S. Moreau.
\end{sphinxadmonition}

\sphinxstepscope


\chapter{Usage}
\label{\detokenize{usage:usage}}\label{\detokenize{usage::doc}}
\sphinxAtStartPar
This page provides the basic usage of the \sphinxcode{\sphinxupquote{amiet\_self\_noise}} package.

\sphinxAtStartPar
\sphinxcode{\sphinxupquote{amiet\_self\_noise}} has been built to be used from a terminal, by providing the inputs thought a combination of command line arguments and a configuration file (written in YAML). This allows to easily run the code with multiple parameters throught simple bash scripts. However, it is also possible to use the package as a library, by importing the \sphinxcode{\sphinxupquote{amiet\_self\_noise}} module in a Python script. This is useful for more complex use cases, such as running the code in a Jupyter notebook or integrating it into a larger Python, or to perform additional pre or postprocessing. For a more detailed description of the available functions, see the \sphinxcode{\sphinxupquote{amiet\_self\_noise}} {\hyperref[\detokenize{modules::doc}]{\sphinxcrossref{\DUrole{doc}{module}}}} documentation.


\section{Input files}
\label{\detokenize{usage:input-files}}\label{\detokenize{usage:target-to-input-files}}
\sphinxAtStartPar
The \sphinxcode{\sphinxupquote{amiet\_self\_noise}} package requires that the user provides all the necessary inputs as a YAML file. This is intented to reduce the necessity to interact directly with the source code itself, but to still allow for a large flexibility in the inputs. The YAML file should contain all the necessary parameters to run the Amiet model, such as the airfoil geometry, the flow conditions, and the desired output parameters.  YAML was developed as a more human\sphinxhyphen{}readable version of the JSON format, but it is read similarly in Python. YAML file entries are composed of key\sphinxhyphen{}value pairs, where the key is a string and the value can be a string, a number, a list, or another dictionary, for example:
\sphinxSetupCaptionForVerbatim{\sphinxcode{\sphinxupquote{config.yaml}}}
\def\sphinxLiteralBlockLabel{\label{\detokenize{usage:id1}}}
\begin{sphinxVerbatim}[commandchars=\\\{\}]
\PYG{n+nt}{L}\PYG{p}{:}\PYG{+w}{ }\PYG{l+lScalar+lScalarPlain}{1.0}\PYG{+w}{ }\PYG{c+c1}{\PYGZsh{} The span length of the airfoil, in meters}
\PYG{n+nt}{U}\PYG{p}{:}\PYG{+w}{ }\PYG{l+lScalar+lScalarPlain}{100.0}\PYG{+w}{ }\PYG{c+c1}{\PYGZsh{} The freestream velocity, in m/s}
\PYG{n+nt}{data\PYGZus{}path}\PYG{p}{:}\PYG{+w}{ }\PYG{l+lScalar+lScalarPlain}{/path/to/data}\PYG{+w}{ }\PYG{c+c1}{\PYGZsh{} The path to the input data files}
\end{sphinxVerbatim}

\sphinxAtStartPar
This file will be converted to a Python dictionary as:

\begin{sphinxVerbatim}[commandchars=\\\{\}]
\PYG{n}{config} \PYG{o}{=} \PYG{p}{\PYGZob{}}
    \PYG{l+s+s1}{\PYGZsq{}}\PYG{l+s+s1}{L}\PYG{l+s+s1}{\PYGZsq{}}\PYG{p}{:} \PYG{l+m+mf}{1.0}\PYG{p}{,}
    \PYG{l+s+s1}{\PYGZsq{}}\PYG{l+s+s1}{U}\PYG{l+s+s1}{\PYGZsq{}}\PYG{p}{:} \PYG{l+m+mf}{100.0}\PYG{p}{,}
    \PYG{l+s+s1}{\PYGZsq{}}\PYG{l+s+s1}{data\PYGZus{}path}\PYG{l+s+s1}{\PYGZsq{}}\PYG{p}{:} \PYG{l+s+s1}{\PYGZsq{}}\PYG{l+s+s1}{/path/to/data}\PYG{l+s+s1}{\PYGZsq{}}
\PYG{p}{\PYGZcb{}}
\end{sphinxVerbatim}

\sphinxAtStartPar
Notice, string do not need to be quoted. The YAML file can also contain comments, which are preceded by a \sphinxtitleref{\#} character. Sub\sphinxhyphen{}dictionaries are identified using the indentation level (usually 2 spaces):
\sphinxSetupCaptionForVerbatim{\sphinxcode{\sphinxupquote{config.yaml}}}
\def\sphinxLiteralBlockLabel{\label{\detokenize{usage:id2}}}
\begin{sphinxVerbatim}[commandchars=\\\{\}]
\PYG{n+nt}{airfoil}\PYG{p}{:}
\PYG{+w}{  }\PYG{n+nt}{name}\PYG{p}{:}\PYG{+w}{ }\PYG{l+lScalar+lScalarPlain}{NACA0012}\PYG{+w}{ }\PYG{c+c1}{\PYGZsh{} The name of the airfoil}
\PYG{+w}{  }\PYG{n+nt}{thickness}\PYG{p}{:}\PYG{+w}{ }\PYG{l+lScalar+lScalarPlain}{0.12}\PYG{+w}{ }\PYG{c+c1}{\PYGZsh{} The thickness of the airfoil, in meters}
\PYG{+w}{  }\PYG{n+nt}{chord}\PYG{p}{:}\PYG{+w}{ }\PYG{l+lScalar+lScalarPlain}{1.0}\PYG{+w}{ }\PYG{c+c1}{\PYGZsh{} The chord length of the airfoil, in meters}
\end{sphinxVerbatim}

\sphinxAtStartPar
which will be converted to:

\begin{sphinxVerbatim}[commandchars=\\\{\}]
\PYG{n}{config} \PYG{o}{=} \PYG{p}{\PYGZob{}}
    \PYG{l+s+s1}{\PYGZsq{}}\PYG{l+s+s1}{airfoil}\PYG{l+s+s1}{\PYGZsq{}}\PYG{p}{:} \PYG{p}{\PYGZob{}}
        \PYG{l+s+s1}{\PYGZsq{}}\PYG{l+s+s1}{name}\PYG{l+s+s1}{\PYGZsq{}}\PYG{p}{:} \PYG{l+s+s1}{\PYGZsq{}}\PYG{l+s+s1}{NACA0012}\PYG{l+s+s1}{\PYGZsq{}}\PYG{p}{,}
        \PYG{l+s+s1}{\PYGZsq{}}\PYG{l+s+s1}{thickness}\PYG{l+s+s1}{\PYGZsq{}}\PYG{p}{:} \PYG{l+m+mf}{0.12}\PYG{p}{,}
        \PYG{l+s+s1}{\PYGZsq{}}\PYG{l+s+s1}{chord}\PYG{l+s+s1}{\PYGZsq{}}\PYG{p}{:} \PYG{l+m+mf}{1.0}
    \PYG{p}{\PYGZcb{}}
\PYG{p}{\PYGZcb{}}
\end{sphinxVerbatim}

\sphinxAtStartPar
Finally, lists are represented using the \sphinxtitleref{\sphinxhyphen{}} character at the begininng of a line, or using square brackets \sphinxtitleref{{[}{]}}:
\sphinxSetupCaptionForVerbatim{\sphinxcode{\sphinxupquote{config.yaml}}}
\def\sphinxLiteralBlockLabel{\label{\detokenize{usage:id3}}}
\begin{sphinxVerbatim}[commandchars=\\\{\}]
\PYG{n+nt}{frequencies}\PYG{p}{:}\PYG{+w}{ }\PYG{p+pIndicator}{[}\PYG{n+nv}{100}\PYG{p+pIndicator}{,}\PYG{+w}{ }\PYG{n+nv}{200}\PYG{p+pIndicator}{,}\PYG{+w}{ }\PYG{n+nv}{300}\PYG{p+pIndicator}{]}
\PYG{n+nt}{angles}\PYG{p}{:}
\PYG{+w}{  }\PYG{p+pIndicator}{\PYGZhy{}}\PYG{+w}{ }\PYG{l+lScalar+lScalarPlain}{0.0}
\PYG{+w}{  }\PYG{p+pIndicator}{\PYGZhy{}}\PYG{+w}{ }\PYG{l+lScalar+lScalarPlain}{5.0}
\PYG{+w}{  }\PYG{p+pIndicator}{\PYGZhy{}}\PYG{+w}{ }\PYG{l+lScalar+lScalarPlain}{10.0}
\end{sphinxVerbatim}

\begin{sphinxVerbatim}[commandchars=\\\{\}]
\PYG{n}{config} \PYG{o}{=} \PYG{p}{\PYGZob{}}
    \PYG{l+s+s1}{\PYGZsq{}}\PYG{l+s+s1}{frequencies}\PYG{l+s+s1}{\PYGZsq{}}\PYG{p}{:} \PYG{p}{[}\PYG{l+m+mi}{100}\PYG{p}{,} \PYG{l+m+mi}{200}\PYG{p}{,} \PYG{l+m+mi}{300}\PYG{p}{]}\PYG{p}{,}
    \PYG{l+s+s1}{\PYGZsq{}}\PYG{l+s+s1}{angles}\PYG{l+s+s1}{\PYGZsq{}}\PYG{p}{:} \PYG{p}{[}\PYG{l+m+mf}{0.0}\PYG{p}{,} \PYG{l+m+mf}{5.0}\PYG{p}{,} \PYG{l+m+mf}{10.0}\PYG{p}{]}
\PYG{p}{\PYGZcb{}}
\end{sphinxVerbatim}

\sphinxAtStartPar
The \sphinxcode{\sphinxupquote{amiet\_self\_noise}} package requires the following keys in the YAML file:
\sphinxSetupCaptionForVerbatim{\sphinxcode{\sphinxupquote{config.yaml}}}
\def\sphinxLiteralBlockLabel{\label{\detokenize{usage:id4}}}
\begin{sphinxVerbatim}[commandchars=\\\{\}]
\PYG{n+nt}{b}\PYG{p}{:}\PYG{+w}{ }\PYG{l+lScalar+lScalarPlain}{1.0}\PYG{+w}{   }\PYG{c+c1}{\PYGZsh{} The airfoil semichord, in meters (float)}
\PYG{n+nt}{L}\PYG{p}{:}\PYG{+w}{ }\PYG{l+lScalar+lScalarPlain}{1.0}\PYG{+w}{   }\PYG{c+c1}{\PYGZsh{} The span length of the airfoil, in meters (float)}
\PYG{n+nt}{T}\PYG{p}{:}\PYG{+w}{ }\PYG{l+lScalar+lScalarPlain}{300.0}\PYG{+w}{ }\PYG{c+c1}{\PYGZsh{} The temperature, in Kelvin (float)}
\PYG{n+nt}{U}\PYG{p}{:}\PYG{+w}{ }\PYG{l+lScalar+lScalarPlain}{100.0}\PYG{+w}{ }\PYG{c+c1}{\PYGZsh{} The freestream velocity, in m/s (float)}
\PYG{c+c1}{\PYGZsh{}}
\PYG{n+nt}{data\PYGZus{}path}\PYG{p}{:}\PYG{+w}{ }\PYG{l+lScalar+lScalarPlain}{/path/to/data.h5}\PYG{+w}{ }\PYG{c+c1}{\PYGZsh{} The path to the data files}
\PYG{n+nt}{data\PYGZus{}type}\PYG{p}{:}\PYG{+w}{ }\PYG{l+lScalar+lScalarPlain}{dns}\PYG{+w}{ }\PYG{c+c1}{\PYGZsh{} Type of the input data, for now can only be \PYGZsq{}dns\PYGZsq{} (string)}
\PYG{n+nt}{mesh\PYGZus{}path}\PYG{p}{:}\PYG{+w}{ }\PYG{l+lScalar+lScalarPlain}{/path/to/mesh.5}\PYG{+w}{  }\PYG{c+c1}{\PYGZsh{} The path to the mesh file, if applicable}
\PYG{c+c1}{\PYGZsh{}}
\PYG{n+nt}{obs}\PYG{p}{:}
\PYG{p+pIndicator}{\PYGZhy{}}\PYG{+w}{ }\PYG{p+pIndicator}{[}\PYG{n+nv}{0.0}\PYG{p+pIndicator}{,}\PYG{+w}{ }\PYG{n+nv}{0.0}\PYG{p+pIndicator}{,}\PYG{+w}{ }\PYG{n+nv}{0.0}\PYG{p+pIndicator}{]}\PYG{+w}{ }\PYG{c+c1}{\PYGZsh{} The cartesian coordinates of the observer,}
\PYG{p+pIndicator}{\PYGZhy{}}\PYG{+w}{ }\PYG{p+pIndicator}{[}\PYG{n+nv}{0.0}\PYG{p+pIndicator}{,}\PYG{+w}{ }\PYG{n+nv}{0.0}\PYG{p+pIndicator}{,}\PYG{+w}{ }\PYG{n+nv}{1.0}\PYG{p+pIndicator}{]}\PYG{+w}{ }\PYG{c+c1}{\PYGZsh{} w.r.t the trailing edge, in meters (list of lists)}
\PYG{c+c1}{\PYGZsh{}}
\PYG{n+nt}{xprobes}\PYG{p}{:}\PYG{+w}{ }\PYG{l+lScalar+lScalarPlain}{0}\PYG{+w}{ }\PYG{c+c1}{\PYGZsh{} The index of the probe in the chord\PYGZhy{}wise direction.}
\PYG{n+nt}{yprobes}\PYG{p}{:}\PYG{+w}{ }\PYG{l+lScalar+lScalarPlain}{0}\PYG{+w}{ }\PYG{c+c1}{\PYGZsh{} The index of the probe in the span\PYGZhy{}wise direction.}
\end{sphinxVerbatim}

\sphinxAtStartPar
The \sphinxcode{\sphinxupquote{xprobes}} and \sphinxcode{\sphinxupquote{yprobes}} are used to select the probes in the input data. They can be a single integer, a list of integers or \sphinxcode{\sphinxupquote{\textquotesingle{}null\textquotesingle{}}}. In the last two cases, they get converted to slice objects. Passing an int will select a probe at that index. Passing a list \sphinxcode{\sphinxupquote{{[}a,b{]}}} will result in secting the probes from \sphinxcode{\sphinxupquote{a}} to \sphinxcode{\sphinxupquote{b}} (\sphinxcode{\sphinxupquote{b}} not included), as in \sphinxcode{\sphinxupquote{np.array{[}a:b{]}}}. Passing \sphinxcode{\sphinxupquote{\textquotesingle{}null\textquotesingle{}}} will select all the probes in that direction, as in \sphinxcode{\sphinxupquote{np.array{[}:{]}}}.

\sphinxstepscope


\chapter{Theoretical reference}
\label{\detokenize{theory:theoretical-reference}}\label{\detokenize{theory::doc}}
\sphinxAtStartPar
This page provides a brief overview of the theoretical background of the Amiet model for trailing\sphinxhyphen{}edge noise. While it is not strictly necessary to know the theory to use tha package, having a clear understading of the underlying physics and assumptions can help in evaluating the results and choosing the appropriate parameters for the model, as well as understading some parts of the documentation. For an in depth description of the model, we advise to refer to the original paper by Amiet (1975) %
\begin{footnote}[1]\sphinxAtStartFootnote
Amiet, R. K. (1975). Acoustic radiation from an airfoil in a
turbulent stream. Journal of Sound and Vibration, 41(4), 407\sphinxhyphen{}420.
%
\end{footnote} and the subsequent extension by Roger \& Moreau (2005) %
\begin{footnote}[2]\sphinxAtStartFootnote
Roger, M., \& Moreau, S. (2005). Back\sphinxhyphen{}scattering correction and
further extensions of Amiet’s trailing\sphinxhyphen{}edge noise model.
Part 1: theory. Journal of Sound and Vibration, 286(3), 477\sphinxhyphen{}506.
%
\end{footnote}.

\sphinxAtStartPar
On a fundamental level, noise is generated at the trailing edge of an airfoil because the pressure fluctuations induced by the turbulent boundary layer get  scattered by the trailing edge. In the original paper, the airfoil is considered as a semi\sphinxhyphen{}infinite flat plate, and a correction to take into account the presence of the leading\sphinxhyphen{}edge was later introduced by Roger \& Moreau (2005). While it is understood that the ultime sound source is the turbulent boundary layer, the Amiet model uses the pressure fluctuations at the trailing edge as a proxy for the sound source. The presence of the trailing edge is then modeled as a linear filter that modifies the pressure fluctuations to produce the sound radiation.

\sphinxAtStartPar
In the simplified form used for the present package, the Amiet model’s prediction of the sound power spectral density at a point in space \(\mathbf{x}\) and frequency \(\omega\) is given by:
\begin{equation}\label{equation:theory:amiet-model}
\begin{split}S_{pp}^{te}(\mathbf{x},\omega) = \left(\frac{\omega x_3 c}{2\pi c_0 S_0^2}\right)^2 L\Phi_{pp}(\omega)\ell_y\left(\omega\right)\left\vert\mathcal L\left(\frac{\omega}{U_c}\right)\right\vert^2\end{split}
\end{equation}
\sphinxAtStartPar
where:
\begin{itemize}
\item {} 
\sphinxAtStartPar
\(c = 2b\) is the airfoil chord (m)

\item {} 
\sphinxAtStartPar
\(c_0\) is the speed of sound (m/s)

\item {} 
\sphinxAtStartPar
\(S_0\) is the corrected observer’s distance (m)

\item {} 
\sphinxAtStartPar
\(L\) is the span of the airfoil (m)

\item {} 
\sphinxAtStartPar
\(\Phi_{pp}\) is the wall pressure fluctuations autospectrum (\(\mathrm{Pa}^2/\mathrm{Hz}\))

\item {} 
\sphinxAtStartPar
\(\ell_y\) is the span\sphinxhyphen{}wise coherence length (m)

\item {} 
\sphinxAtStartPar
\(U_c\) is the turbulent eddies’ convection speed (m/s)

\item {} 
\sphinxAtStartPar
\(\mathcal{L}\) is the radiation integral.

\end{itemize}

\sphinxAtStartPar
References

\sphinxstepscope


\chapter{amiet\sphinxhyphen{}self\sphinxhyphen{}noise modules}
\label{\detokenize{modules:amiet-self-noise-modules}}\label{\detokenize{modules::doc}}
\sphinxstepscope


\section{amiet\_model module}
\label{\detokenize{amiet_model:module-amiet_self_noise.amiet_model}}\label{\detokenize{amiet_model:amiet-model-module}}\label{\detokenize{amiet_model::doc}}\index{module@\spxentry{module}!amiet\_self\_noise.amiet\_model@\spxentry{amiet\_self\_noise.amiet\_model}}\index{amiet\_self\_noise.amiet\_model@\spxentry{amiet\_self\_noise.amiet\_model}!module@\spxentry{module}}\index{AmietModel (class in amiet\_self\_noise.amiet\_model)@\spxentry{AmietModel}\spxextra{class in amiet\_self\_noise.amiet\_model}}

\begin{fulllineitems}
\phantomsection\label{\detokenize{amiet_model:amiet_self_noise.amiet_model.AmietModel}}
\pysigstartsignatures
\pysiglinewithargsret
{\sphinxbfcode{\sphinxupquote{\DUrole{k}{class}\DUrole{w}{ }}}\sphinxcode{\sphinxupquote{amiet\_self\_noise.amiet\_model.}}\sphinxbfcode{\sphinxupquote{AmietModel}}}
{\sphinxparam{\DUrole{n}{input\_data}}}
{}
\pysigstopsignatures
\sphinxAtStartPar
Bases: \sphinxcode{\sphinxupquote{object}}

\sphinxAtStartPar
Amiet model for aeroacoustic noise prediction from airfoils.

\sphinxAtStartPar
The Amiet model is used to predict the noise generated by turbulent
boundary layers on airfoil surfaces. This class implements the complete
Amiet model including wall pressure spectrum computation, coherence
length calculation, and radiation integral evaluation.
\begin{quote}\begin{description}
\sphinxlineitem{Parameters}
\sphinxAtStartPar
\sphinxstyleliteralstrong{\sphinxupquote{input\_data}} (\sphinxstyleliteralemphasis{\sphinxupquote{object}}) \textendash{} Input data object containing pressure measurements, positions,
sampling frequency, and configuration parameters.

\end{description}\end{quote}
\index{input\_data (amiet\_self\_noise.amiet\_model.AmietModel attribute)@\spxentry{input\_data}\spxextra{amiet\_self\_noise.amiet\_model.AmietModel attribute}}

\begin{fulllineitems}
\phantomsection\label{\detokenize{amiet_model:amiet_self_noise.amiet_model.AmietModel.input_data}}
\pysigstartsignatures
\pysigline
{\sphinxbfcode{\sphinxupquote{input\_data}}}
\pysigstopsignatures
\sphinxAtStartPar
Stored input data object containing all necessary parameters
and measurements for the Amiet model computation.
\begin{quote}\begin{description}
\sphinxlineitem{Type}
\sphinxAtStartPar
object

\end{description}\end{quote}

\end{fulllineitems}


\begin{sphinxadmonition}{note}{Note:}
\sphinxAtStartPar
The Amiet model is based on the theoretical framework developed by
Amiet (1975) for predicting airfoil trailing edge noise from
turbulent boundary layer pressure fluctuations.
\end{sphinxadmonition}
\subsubsection*{References}
\index{compute\_coherence() (amiet\_self\_noise.amiet\_model.AmietModel method)@\spxentry{compute\_coherence()}\spxextra{amiet\_self\_noise.amiet\_model.AmietModel method}}

\begin{fulllineitems}
\phantomsection\label{\detokenize{amiet_model:amiet_self_noise.amiet_model.AmietModel.compute_coherence}}
\pysigstartsignatures
\pysiglinewithargsret
{\sphinxbfcode{\sphinxupquote{compute\_coherence}}}
{}
{}
\pysigstopsignatures
\sphinxAtStartPar
Compute the spanwise coherence length from pressure measurements.

\sphinxAtStartPar
This method calculates the spanwise coherence length which
characterizes the correlation of pressure fluctuations across
the airfoil span.
\begin{quote}\begin{description}
\sphinxlineitem{Returns}
\sphinxAtStartPar
\begin{itemize}
\item {} 
\sphinxAtStartPar
\sphinxstylestrong{f} (\sphinxstyleemphasis{ndarray}) \textendash{} Frequency array in Hz, shape (n\_freq,).

\item {} 
\sphinxAtStartPar
\sphinxstylestrong{ly} (\sphinxstyleemphasis{ndarray}) \textendash{} Spanwise coherence length in meters, shape (n\_freq,).

\end{itemize}


\end{description}\end{quote}

\begin{sphinxadmonition}{note}{Note:}
\sphinxAtStartPar
The coherence length is computed using:
\begin{itemize}
\item {} 
\sphinxAtStartPar
Bandpass filtering between 1600\sphinxhyphen{}8000 Hz (2nd order)

\item {} 
\sphinxAtStartPar
Reference position at the middle of the measurement array

\item {} 
\sphinxAtStartPar
Same windowing parameters as wall pressure spectrum

\item {} 
\sphinxAtStartPar
Hanning window with 50\% overlap

\end{itemize}
\end{sphinxadmonition}

\sphinxAtStartPar
The coherence length represents the spanwise extent over which
pressure fluctuations remain correlated.

\end{fulllineitems}

\index{compute\_psd() (amiet\_self\_noise.amiet\_model.AmietModel method)@\spxentry{compute\_psd()}\spxextra{amiet\_self\_noise.amiet\_model.AmietModel method}}

\begin{fulllineitems}
\phantomsection\label{\detokenize{amiet_model:amiet_self_noise.amiet_model.AmietModel.compute_psd}}
\pysigstartsignatures
\pysiglinewithargsret
{\sphinxbfcode{\sphinxupquote{compute\_psd}}}
{}
{}
\pysigstopsignatures
\sphinxAtStartPar
Compute the power spectral density of radiated noise.

\sphinxAtStartPar
This method calculates the complete power spectral density by
combining the wall pressure spectrum, coherence length, and
radiation integral with appropriate directivity corrections.
\begin{quote}\begin{description}
\sphinxlineitem{Returns}
\sphinxAtStartPar
\begin{itemize}
\item {} 
\sphinxAtStartPar
\sphinxstylestrong{f} (\sphinxstyleemphasis{ndarray}) \textendash{} Frequency array in Hz, shape (n\_freq,).

\item {} 
\sphinxAtStartPar
\sphinxstylestrong{psd} (\sphinxstyleemphasis{ndarray}) \textendash{} Power spectral density in Pa\(\sp{\text{2}}\)/Hz, shape (n\_freq, n\_obs).
Each column corresponds to an observer position.

\end{itemize}


\end{description}\end{quote}

\begin{sphinxadmonition}{note}{Note:}
\sphinxAtStartPar
The PSD computation follows the Amiet model formulation:
\begin{equation*}
\begin{split}S_{pp} = D \cdot 2L \cdot \Phi_{pp} \cdot \ell_y \cdot |I|^2\end{split}
\end{equation*}
\sphinxAtStartPar
where \(D\) is the directivity factor, \(L\) is the airfoil length,
\(\Phi_{pp}\) is the wall pressure spectrum, \(\ell_y\) is the coherence length,
and \(I\) is the radiation integral.
\end{sphinxadmonition}

\end{fulllineitems}

\index{compute\_radiation\_integral() (amiet\_self\_noise.amiet\_model.AmietModel method)@\spxentry{compute\_radiation\_integral()}\spxextra{amiet\_self\_noise.amiet\_model.AmietModel method}}

\begin{fulllineitems}
\phantomsection\label{\detokenize{amiet_model:amiet_self_noise.amiet_model.AmietModel.compute_radiation_integral}}
\pysigstartsignatures
\pysiglinewithargsret
{\sphinxbfcode{\sphinxupquote{compute\_radiation\_integral}}}
{\sphinxparam{\DUrole{n}{f}}\sphinxparamcomma \sphinxparam{\DUrole{n}{observer}}}
{}
\pysigstopsignatures
\sphinxAtStartPar
Compute the Amiet radiation integral for a given observer.

\sphinxAtStartPar
The radiation integral accounts for the acoustic scattering
from the airfoil trailing edge and is a key component of
the Amiet model.
\begin{quote}\begin{description}
\sphinxlineitem{Parameters}\begin{itemize}
\item {} 
\sphinxAtStartPar
\sphinxstyleliteralstrong{\sphinxupquote{f}} (\sphinxstyleliteralemphasis{\sphinxupquote{ndarray}}) \textendash{} Frequency array in Hz, shape (n\_freq,).

\item {} 
\sphinxAtStartPar
\sphinxstyleliteralstrong{\sphinxupquote{observer}} (\sphinxstyleliteralemphasis{\sphinxupquote{array\_like}}) \textendash{} Observer position {[}x, y, z{]} in meters, shape (3,).

\end{itemize}

\sphinxlineitem{Returns}
\sphinxAtStartPar
\sphinxstylestrong{I} \textendash{} Complex radiation integral values, shape (n\_freq,).

\sphinxlineitem{Return type}
\sphinxAtStartPar
ndarray, complex

\end{description}\end{quote}

\begin{sphinxadmonition}{note}{Note:}
\sphinxAtStartPar
The radiation integral is computed using:
\begin{itemize}
\item {} 
\sphinxAtStartPar
Observer distance S0 corrected for Mach number effects

\item {} 
\sphinxAtStartPar
Fixed alpha parameter = 0.7 (empirical constant)

\end{itemize}
\end{sphinxadmonition}

\sphinxAtStartPar
The integral represents the acoustic transfer function from
surface pressure fluctuations to far\sphinxhyphen{}field sound pressure.

\begin{sphinxadmonition}{warning}{Warning:}
\sphinxAtStartPar
The current version only supports constant convection velocity, fixed to
0.7 of the freestream velocity.
\end{sphinxadmonition}

\end{fulllineitems}

\index{compute\_wps() (amiet\_self\_noise.amiet\_model.AmietModel method)@\spxentry{compute\_wps()}\spxextra{amiet\_self\_noise.amiet\_model.AmietModel method}}

\begin{fulllineitems}
\phantomsection\label{\detokenize{amiet_model:amiet_self_noise.amiet_model.AmietModel.compute_wps}}
\pysigstartsignatures
\pysiglinewithargsret
{\sphinxbfcode{\sphinxupquote{compute\_wps}}}
{}
{}
\pysigstopsignatures
\sphinxAtStartPar
Compute the wall pressure spectrum from pressure measurements.

\sphinxAtStartPar
This method calculates the wall pressure spectrum using Welch’s
method with Hanning window and 50\% overlap.
\begin{quote}\begin{description}
\sphinxlineitem{Returns}
\sphinxAtStartPar
\begin{itemize}
\item {} 
\sphinxAtStartPar
\sphinxstylestrong{f} (\sphinxstyleemphasis{ndarray}) \textendash{} Frequency array in Hz, shape (n\_freq,).

\item {} 
\sphinxAtStartPar
\sphinxstylestrong{phi\_pp} (\sphinxstyleemphasis{ndarray}) \textendash{} Wall pressure spectrum in Pa\(\sp{\text{2}}\)·s, shape (n\_freq,).

\end{itemize}


\end{description}\end{quote}

\begin{sphinxadmonition}{note}{Note:}
\sphinxAtStartPar
The spectrum is computed using:
\begin{itemize}
\item {} 
\sphinxAtStartPar
Segment length: N/8 samples (where N is total number of samples)

\item {} 
\sphinxAtStartPar
Overlap: 50\% of segment length

\item {} 
\sphinxAtStartPar
Window: Hanning window

\item {} 
\sphinxAtStartPar
No additional filtering is applied (filter=False)

\end{itemize}
\end{sphinxadmonition}

\end{fulllineitems}


\end{fulllineitems}


\sphinxstepscope


\section{preproc module}
\label{\detokenize{preproc:preproc-module}}\label{\detokenize{preproc::doc}}
\sphinxAtStartPar
Pre\sphinxhyphen{}process data to compute the relevant statistics (mainly the pressure spectrum and the correlation length).
\index{module@\spxentry{module}!amiet\_self\_noise.preproc@\spxentry{amiet\_self\_noise.preproc}}\index{amiet\_self\_noise.preproc@\spxentry{amiet\_self\_noise.preproc}!module@\spxentry{module}}\index{coherence\_function() (in module amiet\_self\_noise.preproc)@\spxentry{coherence\_function()}\spxextra{in module amiet\_self\_noise.preproc}}\phantomsection\label{\detokenize{preproc:module-amiet_self_noise.preproc}}

\begin{fulllineitems}
\phantomsection\label{\detokenize{preproc:amiet_self_noise.preproc.coherence_function}}
\pysigstartsignatures
\pysiglinewithargsret
{\sphinxcode{\sphinxupquote{amiet\_self\_noise.preproc.}}\sphinxbfcode{\sphinxupquote{coherence\_function}}}
{\sphinxparam{\DUrole{n}{data}\DUrole{p}{:}\DUrole{w}{ }\DUrole{n}{ndarray}}\sphinxparamcomma \sphinxparam{\DUrole{n}{ref\_index}\DUrole{p}{:}\DUrole{w}{ }\DUrole{n}{int}\DUrole{w}{ }\DUrole{o}{=}\DUrole{w}{ }\DUrole{default_value}{0}}\sphinxparamcomma \sphinxparam{\DUrole{n}{filter}\DUrole{p}{:}\DUrole{w}{ }\DUrole{n}{bool}\DUrole{w}{ }\DUrole{o}{=}\DUrole{w}{ }\DUrole{default_value}{False}}\sphinxparamcomma \sphinxparam{\DUrole{n}{flims}\DUrole{p}{:}\DUrole{w}{ }\DUrole{n}{tuple}\DUrole{w}{ }\DUrole{o}{=}\DUrole{w}{ }\DUrole{default_value}{(0.0, 1.0)}}\sphinxparamcomma \sphinxparam{\DUrole{n}{fs}\DUrole{p}{:}\DUrole{w}{ }\DUrole{n}{float}\DUrole{w}{ }\DUrole{o}{=}\DUrole{w}{ }\DUrole{default_value}{1.0}}\sphinxparamcomma \sphinxparam{\DUrole{n}{order}\DUrole{p}{:}\DUrole{w}{ }\DUrole{n}{int}\DUrole{w}{ }\DUrole{o}{=}\DUrole{w}{ }\DUrole{default_value}{2}}\sphinxparamcomma \sphinxparam{\DUrole{o}{**}\DUrole{n}{kwargs}}}
{}
\pysigstopsignatures
\sphinxAtStartPar
Compute the coherence function for the input data.
\begin{quote}\begin{description}
\sphinxlineitem{Parameters}\begin{itemize}
\item {} 
\sphinxAtStartPar
\sphinxstyleliteralstrong{\sphinxupquote{data}} (\sphinxstyleliteralemphasis{\sphinxupquote{np.ndarray}}) \textendash{} Input data array where each column represents a sensor.

\item {} 
\sphinxAtStartPar
\sphinxstyleliteralstrong{\sphinxupquote{ref\_index}} (\sphinxstyleliteralemphasis{\sphinxupquote{int}}\sphinxstyleliteralemphasis{\sphinxupquote{, }}\sphinxstyleliteralemphasis{\sphinxupquote{optional}}) \textendash{} Index of the reference sensor in the data array. Default is 0.

\item {} 
\sphinxAtStartPar
\sphinxstyleliteralstrong{\sphinxupquote{filter}} (\sphinxstyleliteralemphasis{\sphinxupquote{bool}}\sphinxstyleliteralemphasis{\sphinxupquote{, }}\sphinxstyleliteralemphasis{\sphinxupquote{optional}}) \textendash{} \begin{description}
\sphinxlineitem{If True, apply a bandpass filter to the data before computing the}
\sphinxAtStartPar
coherence. Default is False.

\end{description}


\item {} 
\sphinxAtStartPar
\sphinxstyleliteralstrong{\sphinxupquote{flims}} (\sphinxstyleliteralemphasis{\sphinxupquote{tuple}}\sphinxstyleliteralemphasis{\sphinxupquote{, }}\sphinxstyleliteralemphasis{\sphinxupquote{optional}}) \textendash{} \begin{description}
\sphinxlineitem{Frequency limits for the bandpass filter (low, high) in Hz. Default is}
\sphinxAtStartPar
(0.0, 1.0).

\end{description}


\item {} 
\sphinxAtStartPar
\sphinxstyleliteralstrong{\sphinxupquote{fs}} (\sphinxstyleliteralemphasis{\sphinxupquote{float}}\sphinxstyleliteralemphasis{\sphinxupquote{, }}\sphinxstyleliteralemphasis{\sphinxupquote{optional}}) \textendash{} Sampling frequency in Hz. Default is 1.0.

\item {} 
\sphinxAtStartPar
\sphinxstyleliteralstrong{\sphinxupquote{order}} (\sphinxstyleliteralemphasis{\sphinxupquote{int}}\sphinxstyleliteralemphasis{\sphinxupquote{, }}\sphinxstyleliteralemphasis{\sphinxupquote{optional}}) \textendash{} Order of the Butterworth filter. Default is 2.

\item {} 
\sphinxAtStartPar
\sphinxstyleliteralstrong{\sphinxupquote{**kwargs}} (\sphinxstyleliteralemphasis{\sphinxupquote{dict}}\sphinxstyleliteralemphasis{\sphinxupquote{, }}\sphinxstyleliteralemphasis{\sphinxupquote{optional}}) \textendash{} Additional keyword arguments passed to \sphinxtitleref{scipy.signal.coherence}.

\end{itemize}

\sphinxlineitem{Returns}
\sphinxAtStartPar
\begin{itemize}
\item {} 
\sphinxAtStartPar
\sphinxstylestrong{f} (\sphinxstyleemphasis{np.ndarray}) \textendash{} Frequencies at which the coherence is computed.

\item {} 
\sphinxAtStartPar
\sphinxstylestrong{gamma} (\sphinxstyleemphasis{np.ndarray}) \textendash{} Coherence values for each sensor with respect to the reference sensor.

\end{itemize}


\end{description}\end{quote}

\end{fulllineitems}

\index{coherence\_length() (in module amiet\_self\_noise.preproc)@\spxentry{coherence\_length()}\spxextra{in module amiet\_self\_noise.preproc}}

\begin{fulllineitems}
\phantomsection\label{\detokenize{preproc:amiet_self_noise.preproc.coherence_length}}
\pysigstartsignatures
\pysiglinewithargsret
{\sphinxcode{\sphinxupquote{amiet\_self\_noise.preproc.}}\sphinxbfcode{\sphinxupquote{coherence\_length}}}
{\sphinxparam{\DUrole{n}{data}\DUrole{p}{:}\DUrole{w}{ }\DUrole{n}{ndarray}}\sphinxparamcomma \sphinxparam{\DUrole{n}{z}\DUrole{p}{:}\DUrole{w}{ }\DUrole{n}{ndarray}}\sphinxparamcomma \sphinxparam{\DUrole{n}{ref\_index}\DUrole{p}{:}\DUrole{w}{ }\DUrole{n}{int}\DUrole{w}{ }\DUrole{o}{=}\DUrole{w}{ }\DUrole{default_value}{0}}\sphinxparamcomma \sphinxparam{\DUrole{n}{filter}\DUrole{p}{:}\DUrole{w}{ }\DUrole{n}{bool}\DUrole{w}{ }\DUrole{o}{=}\DUrole{w}{ }\DUrole{default_value}{False}}\sphinxparamcomma \sphinxparam{\DUrole{n}{flims}\DUrole{p}{:}\DUrole{w}{ }\DUrole{n}{tuple}\DUrole{w}{ }\DUrole{o}{=}\DUrole{w}{ }\DUrole{default_value}{(0.0, 1.0)}}\sphinxparamcomma \sphinxparam{\DUrole{n}{fs}\DUrole{p}{:}\DUrole{w}{ }\DUrole{n}{float}\DUrole{w}{ }\DUrole{o}{=}\DUrole{w}{ }\DUrole{default_value}{1.0}}\sphinxparamcomma \sphinxparam{\DUrole{n}{order}\DUrole{p}{:}\DUrole{w}{ }\DUrole{n}{int}\DUrole{w}{ }\DUrole{o}{=}\DUrole{w}{ }\DUrole{default_value}{2}}\sphinxparamcomma \sphinxparam{\DUrole{o}{**}\DUrole{n}{kwargs}}}
{}
\pysigstopsignatures
\sphinxAtStartPar
Compute the coherence function for the input data.
\begin{quote}\begin{description}
\sphinxlineitem{Parameters}\begin{itemize}
\item {} 
\sphinxAtStartPar
\sphinxstyleliteralstrong{\sphinxupquote{data}} (\sphinxstyleliteralemphasis{\sphinxupquote{np.ndarray}}) \textendash{} Input data array where each column represents a sensor.

\item {} 
\sphinxAtStartPar
\sphinxstyleliteralstrong{\sphinxupquote{z}} (\sphinxstyleliteralemphasis{\sphinxupquote{np.ndarray}}) \textendash{} Array of sensor indices or positions corresponding to the data columns.

\item {} 
\sphinxAtStartPar
\sphinxstyleliteralstrong{\sphinxupquote{ref\_index}} (\sphinxstyleliteralemphasis{\sphinxupquote{int}}\sphinxstyleliteralemphasis{\sphinxupquote{, }}\sphinxstyleliteralemphasis{\sphinxupquote{optional}}) \textendash{} Index of the reference sensor in the data array. Default is 0.

\item {} 
\sphinxAtStartPar
\sphinxstyleliteralstrong{\sphinxupquote{filter}} (\sphinxstyleliteralemphasis{\sphinxupquote{bool}}\sphinxstyleliteralemphasis{\sphinxupquote{, }}\sphinxstyleliteralemphasis{\sphinxupquote{optional}}) \textendash{} \begin{description}
\sphinxlineitem{If True, apply a bandpass filter to the data before computing the}
\sphinxAtStartPar
coherence. Default is False.

\end{description}


\item {} 
\sphinxAtStartPar
\sphinxstyleliteralstrong{\sphinxupquote{flims}} (\sphinxstyleliteralemphasis{\sphinxupquote{tuple}}\sphinxstyleliteralemphasis{\sphinxupquote{, }}\sphinxstyleliteralemphasis{\sphinxupquote{optional}}) \textendash{} \begin{description}
\sphinxlineitem{Frequency limits for the bandpass filter (low, high) in Hz. Default is}
\sphinxAtStartPar
(0.0, 1.0).

\end{description}


\item {} 
\sphinxAtStartPar
\sphinxstyleliteralstrong{\sphinxupquote{fs}} (\sphinxstyleliteralemphasis{\sphinxupquote{float}}\sphinxstyleliteralemphasis{\sphinxupquote{, }}\sphinxstyleliteralemphasis{\sphinxupquote{optional}}) \textendash{} Sampling frequency in Hz. Default is 1.0.

\item {} 
\sphinxAtStartPar
\sphinxstyleliteralstrong{\sphinxupquote{order}} (\sphinxstyleliteralemphasis{\sphinxupquote{int}}\sphinxstyleliteralemphasis{\sphinxupquote{, }}\sphinxstyleliteralemphasis{\sphinxupquote{optional}}) \textendash{} Order of the Butterworth filter. Default is 2.

\item {} 
\sphinxAtStartPar
\sphinxstyleliteralstrong{\sphinxupquote{**kwargs}} (\sphinxstyleliteralemphasis{\sphinxupquote{dict}}\sphinxstyleliteralemphasis{\sphinxupquote{, }}\sphinxstyleliteralemphasis{\sphinxupquote{optional}}) \textendash{} Additional keyword arguments passed to \sphinxtitleref{scipy.signal.coherence}.

\end{itemize}

\sphinxlineitem{Returns}
\sphinxAtStartPar
\begin{itemize}
\item {} 
\sphinxAtStartPar
\sphinxstylestrong{f} (\sphinxstyleemphasis{np.ndarray}) \textendash{} Frequencies at which the coherence is computed.

\item {} 
\sphinxAtStartPar
\sphinxstylestrong{lz} (\sphinxstyleemphasis{np.ndarray}) \textendash{} Coherence length at all frequencies.

\end{itemize}


\end{description}\end{quote}

\end{fulllineitems}

\index{spectrum() (in module amiet\_self\_noise.preproc)@\spxentry{spectrum()}\spxextra{in module amiet\_self\_noise.preproc}}

\begin{fulllineitems}
\phantomsection\label{\detokenize{preproc:amiet_self_noise.preproc.spectrum}}
\pysigstartsignatures
\pysiglinewithargsret
{\sphinxcode{\sphinxupquote{amiet\_self\_noise.preproc.}}\sphinxbfcode{\sphinxupquote{spectrum}}}
{\sphinxparam{\DUrole{n}{data}}\sphinxparamcomma \sphinxparam{\DUrole{n}{filter}\DUrole{p}{:}\DUrole{w}{ }\DUrole{n}{bool}\DUrole{w}{ }\DUrole{o}{=}\DUrole{w}{ }\DUrole{default_value}{False}}\sphinxparamcomma \sphinxparam{\DUrole{n}{flims}\DUrole{p}{:}\DUrole{w}{ }\DUrole{n}{tuple}\DUrole{w}{ }\DUrole{o}{=}\DUrole{w}{ }\DUrole{default_value}{(0.0, 1.0)}}\sphinxparamcomma \sphinxparam{\DUrole{n}{fs}\DUrole{p}{:}\DUrole{w}{ }\DUrole{n}{float}\DUrole{w}{ }\DUrole{o}{=}\DUrole{w}{ }\DUrole{default_value}{1.0}}\sphinxparamcomma \sphinxparam{\DUrole{n}{order}\DUrole{p}{:}\DUrole{w}{ }\DUrole{n}{int}\DUrole{w}{ }\DUrole{o}{=}\DUrole{w}{ }\DUrole{default_value}{2}}\sphinxparamcomma \sphinxparam{\DUrole{n}{avg}\DUrole{p}{:}\DUrole{w}{ }\DUrole{n}{int\DUrole{w}{ }\DUrole{p}{|}\DUrole{w}{ }None}\DUrole{w}{ }\DUrole{o}{=}\DUrole{w}{ }\DUrole{default_value}{None}}\sphinxparamcomma \sphinxparam{\DUrole{o}{**}\DUrole{n}{kwargs}}}
{}
\pysigstopsignatures
\sphinxAtStartPar
Compute the power spectral density (PSD) of the input data.
\begin{quote}\begin{description}
\sphinxlineitem{Parameters}\begin{itemize}
\item {} 
\sphinxAtStartPar
\sphinxstyleliteralstrong{\sphinxupquote{data}} (\sphinxstyleliteralemphasis{\sphinxupquote{np.ndarray}}) \textendash{} Input data to compute the spectrum.

\item {} 
\sphinxAtStartPar
\sphinxstyleliteralstrong{\sphinxupquote{filter}} (\sphinxstyleliteralemphasis{\sphinxupquote{bool}}\sphinxstyleliteralemphasis{\sphinxupquote{, }}\sphinxstyleliteralemphasis{\sphinxupquote{optional}}) \textendash{} \begin{description}
\sphinxlineitem{If True, apply a bandpass filter to the data before computing the}
\sphinxAtStartPar
spectrum. Default is False.

\end{description}


\item {} 
\sphinxAtStartPar
\sphinxstyleliteralstrong{\sphinxupquote{flims}} (\sphinxstyleliteralemphasis{\sphinxupquote{tuple}}\sphinxstyleliteralemphasis{\sphinxupquote{, }}\sphinxstyleliteralemphasis{\sphinxupquote{optional}}) \textendash{} \begin{description}
\sphinxlineitem{Frequency limits for the bandpass filter (low, high) in Hz. Default is}
\sphinxAtStartPar
(0.0, 1.0).

\end{description}


\item {} 
\sphinxAtStartPar
\sphinxstyleliteralstrong{\sphinxupquote{fs}} (\sphinxstyleliteralemphasis{\sphinxupquote{float}}\sphinxstyleliteralemphasis{\sphinxupquote{, }}\sphinxstyleliteralemphasis{\sphinxupquote{optional}}) \textendash{} Sampling frequency in Hz. Default is 1.0.

\item {} 
\sphinxAtStartPar
\sphinxstyleliteralstrong{\sphinxupquote{order}} (\sphinxstyleliteralemphasis{\sphinxupquote{int}}\sphinxstyleliteralemphasis{\sphinxupquote{, }}\sphinxstyleliteralemphasis{\sphinxupquote{optional}}) \textendash{} Order of the Butterworth filter. Default is 2.

\item {} 
\sphinxAtStartPar
\sphinxstyleliteralstrong{\sphinxupquote{avg}} (\sphinxstyleliteralemphasis{\sphinxupquote{int}}\sphinxstyleliteralemphasis{\sphinxupquote{, }}\sphinxstyleliteralemphasis{\sphinxupquote{optional}}) \textendash{} Axis along which to average the power spectral density. If None, no
averaging is performed. Default is None.

\item {} 
\sphinxAtStartPar
\sphinxstyleliteralstrong{\sphinxupquote{**kwargs}} (\sphinxstyleliteralemphasis{\sphinxupquote{dict}}\sphinxstyleliteralemphasis{\sphinxupquote{, }}\sphinxstyleliteralemphasis{\sphinxupquote{optional}}) \textendash{} Additional keyword arguments passed to \sphinxtitleref{scipy.signal.welch}.

\end{itemize}

\sphinxlineitem{Returns}
\sphinxAtStartPar
\begin{itemize}
\item {} 
\sphinxAtStartPar
\sphinxstylestrong{f} (\sphinxstyleemphasis{np.ndarray}) \textendash{} Frequencies at which the PSD is computed.

\item {} 
\sphinxAtStartPar
\sphinxstylestrong{spp} (\sphinxstyleemphasis{np.ndarray}) \textendash{} Power spectral density of the input data.

\end{itemize}


\end{description}\end{quote}

\end{fulllineitems}


\sphinxstepscope


\section{io\_utils module}
\label{\detokenize{io:io-utils-module}}\label{\detokenize{io::doc}}
\sphinxAtStartPar
The \sphinxcode{\sphinxupquote{io\_utils}} module is responsible for all the input\sphinxhyphen{}output operations in the \sphinxcode{\sphinxupquote{amiet\_self\_noise}} package.
\index{module@\spxentry{module}!amiet\_self\_noise.io\_utils@\spxentry{amiet\_self\_noise.io\_utils}}\index{amiet\_self\_noise.io\_utils@\spxentry{amiet\_self\_noise.io\_utils}!module@\spxentry{module}}\index{ConfigData (class in amiet\_self\_noise.io\_utils)@\spxentry{ConfigData}\spxextra{class in amiet\_self\_noise.io\_utils}}\phantomsection\label{\detokenize{io:module-amiet_self_noise.io_utils}}

\begin{fulllineitems}
\phantomsection\label{\detokenize{io:amiet_self_noise.io_utils.ConfigData}}
\pysigstartsignatures
\pysiglinewithargsret
{\sphinxbfcode{\sphinxupquote{\DUrole{k}{class}\DUrole{w}{ }}}\sphinxcode{\sphinxupquote{amiet\_self\_noise.io\_utils.}}\sphinxbfcode{\sphinxupquote{ConfigData}}}
{\sphinxparam{\DUrole{n}{b}\DUrole{p}{:}\DUrole{w}{ }\DUrole{n}{float}}\sphinxparamcomma \sphinxparam{\DUrole{n}{T}\DUrole{p}{:}\DUrole{w}{ }\DUrole{n}{float}}\sphinxparamcomma \sphinxparam{\DUrole{n}{L}\DUrole{p}{:}\DUrole{w}{ }\DUrole{n}{float}}\sphinxparamcomma \sphinxparam{\DUrole{n}{rho}\DUrole{p}{:}\DUrole{w}{ }\DUrole{n}{float}}\sphinxparamcomma \sphinxparam{\DUrole{n}{obs}\DUrole{p}{:}\DUrole{w}{ }\DUrole{n}{array}}\sphinxparamcomma \sphinxparam{\DUrole{n}{U0}\DUrole{p}{:}\DUrole{w}{ }\DUrole{n}{float}}\sphinxparamcomma \sphinxparam{\DUrole{n}{data\_type}\DUrole{p}{:}\DUrole{w}{ }\DUrole{n}{str}}\sphinxparamcomma \sphinxparam{\DUrole{n}{data\_path}\DUrole{p}{:}\DUrole{w}{ }\DUrole{n}{str\DUrole{w}{ }\DUrole{p}{|}\DUrole{w}{ }None}\DUrole{w}{ }\DUrole{o}{=}\DUrole{w}{ }\DUrole{default_value}{None}}\sphinxparamcomma \sphinxparam{\DUrole{n}{mesh\_path}\DUrole{p}{:}\DUrole{w}{ }\DUrole{n}{str\DUrole{w}{ }\DUrole{p}{|}\DUrole{w}{ }None}\DUrole{w}{ }\DUrole{o}{=}\DUrole{w}{ }\DUrole{default_value}{None}}\sphinxparamcomma \sphinxparam{\DUrole{n}{out\_dir}\DUrole{p}{:}\DUrole{w}{ }\DUrole{n}{str\DUrole{w}{ }\DUrole{p}{|}\DUrole{w}{ }None}\DUrole{w}{ }\DUrole{o}{=}\DUrole{w}{ }\DUrole{default_value}{None}}\sphinxparamcomma \sphinxparam{\DUrole{n}{xprobes}\DUrole{p}{:}\DUrole{w}{ }\DUrole{n}{int\DUrole{w}{ }\DUrole{p}{|}\DUrole{w}{ }None}\DUrole{w}{ }\DUrole{o}{=}\DUrole{w}{ }\DUrole{default_value}{None}}\sphinxparamcomma \sphinxparam{\DUrole{n}{yprobes}\DUrole{p}{:}\DUrole{w}{ }\DUrole{n}{int\DUrole{w}{ }\DUrole{p}{|}\DUrole{w}{ }None}\DUrole{w}{ }\DUrole{o}{=}\DUrole{w}{ }\DUrole{default_value}{None}}}
{}
\pysigstopsignatures
\sphinxAtStartPar
Bases: \sphinxcode{\sphinxupquote{object}}

\sphinxAtStartPar
Class to hold input data.
\begin{quote}\begin{description}
\sphinxlineitem{Parameters}\begin{itemize}
\item {} 
\sphinxAtStartPar
\sphinxstyleliteralstrong{\sphinxupquote{b}} (\sphinxstyleliteralemphasis{\sphinxupquote{float}}) \textendash{} Half the chord of the airfoil, in meters.

\item {} 
\sphinxAtStartPar
\sphinxstyleliteralstrong{\sphinxupquote{T}} (\sphinxstyleliteralemphasis{\sphinxupquote{float}}) \textendash{} The temperature in Kelvin.

\item {} 
\sphinxAtStartPar
\sphinxstyleliteralstrong{\sphinxupquote{L}} (\sphinxstyleliteralemphasis{\sphinxupquote{float}}) \textendash{} The span of the airfoil, in meters.

\item {} 
\sphinxAtStartPar
\sphinxstyleliteralstrong{\sphinxupquote{rho}} (\sphinxstyleliteralemphasis{\sphinxupquote{float}}) \textendash{} The density of the fluid, in kg/m\textasciicircum{}3.

\item {} 
\sphinxAtStartPar
\sphinxstyleliteralstrong{\sphinxupquote{obs}} (\sphinxstyleliteralemphasis{\sphinxupquote{np.array}}) \textendash{} Observers location in cartesian coordinates, as a numpy array.

\item {} 
\sphinxAtStartPar
\sphinxstyleliteralstrong{\sphinxupquote{U0}} (\sphinxstyleliteralemphasis{\sphinxupquote{float}}) \textendash{} The free stream velocity, in meters per second.

\item {} 
\sphinxAtStartPar
\sphinxstyleliteralstrong{\sphinxupquote{data\_type}} (\sphinxstyleliteralemphasis{\sphinxupquote{str}}) \textendash{} The type of data. Currently only ‘dns’ is supported.

\item {} 
\sphinxAtStartPar
\sphinxstyleliteralstrong{\sphinxupquote{data\_path}} (\sphinxstyleliteralemphasis{\sphinxupquote{str}}) \textendash{} The path to the data file.

\end{itemize}

\end{description}\end{quote}
\index{L (amiet\_self\_noise.io\_utils.ConfigData attribute)@\spxentry{L}\spxextra{amiet\_self\_noise.io\_utils.ConfigData attribute}}

\begin{fulllineitems}
\phantomsection\label{\detokenize{io:amiet_self_noise.io_utils.ConfigData.L}}
\pysigstartsignatures
\pysigline
{\sphinxbfcode{\sphinxupquote{L}}\sphinxbfcode{\sphinxupquote{\DUrole{p}{:}\DUrole{w}{ }float}}}
\pysigstopsignatures
\end{fulllineitems}

\index{M0 (amiet\_self\_noise.io\_utils.ConfigData attribute)@\spxentry{M0}\spxextra{amiet\_self\_noise.io\_utils.ConfigData attribute}}

\begin{fulllineitems}
\phantomsection\label{\detokenize{io:amiet_self_noise.io_utils.ConfigData.M0}}
\pysigstartsignatures
\pysigline
{\sphinxbfcode{\sphinxupquote{M0}}\sphinxbfcode{\sphinxupquote{\DUrole{p}{:}\DUrole{w}{ }float}}}
\pysigstopsignatures
\sphinxAtStartPar
Mach number relative to the free stream velocity

\end{fulllineitems}

\index{T (amiet\_self\_noise.io\_utils.ConfigData attribute)@\spxentry{T}\spxextra{amiet\_self\_noise.io\_utils.ConfigData attribute}}

\begin{fulllineitems}
\phantomsection\label{\detokenize{io:amiet_self_noise.io_utils.ConfigData.T}}
\pysigstartsignatures
\pysigline
{\sphinxbfcode{\sphinxupquote{T}}\sphinxbfcode{\sphinxupquote{\DUrole{p}{:}\DUrole{w}{ }float}}}
\pysigstopsignatures
\end{fulllineitems}

\index{U0 (amiet\_self\_noise.io\_utils.ConfigData attribute)@\spxentry{U0}\spxextra{amiet\_self\_noise.io\_utils.ConfigData attribute}}

\begin{fulllineitems}
\phantomsection\label{\detokenize{io:amiet_self_noise.io_utils.ConfigData.U0}}
\pysigstartsignatures
\pysigline
{\sphinxbfcode{\sphinxupquote{U0}}\sphinxbfcode{\sphinxupquote{\DUrole{p}{:}\DUrole{w}{ }float}}}
\pysigstopsignatures
\end{fulllineitems}

\index{b (amiet\_self\_noise.io\_utils.ConfigData attribute)@\spxentry{b}\spxextra{amiet\_self\_noise.io\_utils.ConfigData attribute}}

\begin{fulllineitems}
\phantomsection\label{\detokenize{io:amiet_self_noise.io_utils.ConfigData.b}}
\pysigstartsignatures
\pysigline
{\sphinxbfcode{\sphinxupquote{b}}\sphinxbfcode{\sphinxupquote{\DUrole{p}{:}\DUrole{w}{ }float}}}
\pysigstopsignatures
\end{fulllineitems}

\index{c0 (amiet\_self\_noise.io\_utils.ConfigData attribute)@\spxentry{c0}\spxextra{amiet\_self\_noise.io\_utils.ConfigData attribute}}

\begin{fulllineitems}
\phantomsection\label{\detokenize{io:amiet_self_noise.io_utils.ConfigData.c0}}
\pysigstartsignatures
\pysigline
{\sphinxbfcode{\sphinxupquote{c0}}\sphinxbfcode{\sphinxupquote{\DUrole{p}{:}\DUrole{w}{ }float}}}
\pysigstopsignatures
\sphinxAtStartPar
The speed of sound at the given temperature, calculated as \(\sqrt{\gamma R T}\)
where \(R=287.05\;\mathrm{J\cdot kg^{-1}\cdot K^{-1}}\) and \(\gamma=1.4\).

\end{fulllineitems}

\index{data\_path (amiet\_self\_noise.io\_utils.ConfigData attribute)@\spxentry{data\_path}\spxextra{amiet\_self\_noise.io\_utils.ConfigData attribute}}

\begin{fulllineitems}
\phantomsection\label{\detokenize{io:amiet_self_noise.io_utils.ConfigData.data_path}}
\pysigstartsignatures
\pysigline
{\sphinxbfcode{\sphinxupquote{data\_path}}\sphinxbfcode{\sphinxupquote{\DUrole{p}{:}\DUrole{w}{ }str\DUrole{w}{ }\DUrole{p}{|}\DUrole{w}{ }None}}\sphinxbfcode{\sphinxupquote{\DUrole{w}{ }\DUrole{p}{=}\DUrole{w}{ }None}}}
\pysigstopsignatures
\end{fulllineitems}

\index{data\_type (amiet\_self\_noise.io\_utils.ConfigData attribute)@\spxentry{data\_type}\spxextra{amiet\_self\_noise.io\_utils.ConfigData attribute}}

\begin{fulllineitems}
\phantomsection\label{\detokenize{io:amiet_self_noise.io_utils.ConfigData.data_type}}
\pysigstartsignatures
\pysigline
{\sphinxbfcode{\sphinxupquote{data\_type}}\sphinxbfcode{\sphinxupquote{\DUrole{p}{:}\DUrole{w}{ }str}}}
\pysigstopsignatures
\end{fulllineitems}

\index{mesh\_path (amiet\_self\_noise.io\_utils.ConfigData attribute)@\spxentry{mesh\_path}\spxextra{amiet\_self\_noise.io\_utils.ConfigData attribute}}

\begin{fulllineitems}
\phantomsection\label{\detokenize{io:amiet_self_noise.io_utils.ConfigData.mesh_path}}
\pysigstartsignatures
\pysigline
{\sphinxbfcode{\sphinxupquote{mesh\_path}}\sphinxbfcode{\sphinxupquote{\DUrole{p}{:}\DUrole{w}{ }str\DUrole{w}{ }\DUrole{p}{|}\DUrole{w}{ }None}}\sphinxbfcode{\sphinxupquote{\DUrole{w}{ }\DUrole{p}{=}\DUrole{w}{ }None}}}
\pysigstopsignatures
\end{fulllineitems}

\index{obs (amiet\_self\_noise.io\_utils.ConfigData attribute)@\spxentry{obs}\spxextra{amiet\_self\_noise.io\_utils.ConfigData attribute}}

\begin{fulllineitems}
\phantomsection\label{\detokenize{io:amiet_self_noise.io_utils.ConfigData.obs}}
\pysigstartsignatures
\pysigline
{\sphinxbfcode{\sphinxupquote{obs}}\sphinxbfcode{\sphinxupquote{\DUrole{p}{:}\DUrole{w}{ }array}}}
\pysigstopsignatures
\end{fulllineitems}

\index{out\_dir (amiet\_self\_noise.io\_utils.ConfigData attribute)@\spxentry{out\_dir}\spxextra{amiet\_self\_noise.io\_utils.ConfigData attribute}}

\begin{fulllineitems}
\phantomsection\label{\detokenize{io:amiet_self_noise.io_utils.ConfigData.out_dir}}
\pysigstartsignatures
\pysigline
{\sphinxbfcode{\sphinxupquote{out\_dir}}\sphinxbfcode{\sphinxupquote{\DUrole{p}{:}\DUrole{w}{ }str\DUrole{w}{ }\DUrole{p}{|}\DUrole{w}{ }None}}\sphinxbfcode{\sphinxupquote{\DUrole{w}{ }\DUrole{p}{=}\DUrole{w}{ }None}}}
\pysigstopsignatures
\end{fulllineitems}

\index{rho (amiet\_self\_noise.io\_utils.ConfigData attribute)@\spxentry{rho}\spxextra{amiet\_self\_noise.io\_utils.ConfigData attribute}}

\begin{fulllineitems}
\phantomsection\label{\detokenize{io:amiet_self_noise.io_utils.ConfigData.rho}}
\pysigstartsignatures
\pysigline
{\sphinxbfcode{\sphinxupquote{rho}}\sphinxbfcode{\sphinxupquote{\DUrole{p}{:}\DUrole{w}{ }float}}}
\pysigstopsignatures
\end{fulllineitems}

\index{xprobes (amiet\_self\_noise.io\_utils.ConfigData attribute)@\spxentry{xprobes}\spxextra{amiet\_self\_noise.io\_utils.ConfigData attribute}}

\begin{fulllineitems}
\phantomsection\label{\detokenize{io:amiet_self_noise.io_utils.ConfigData.xprobes}}
\pysigstartsignatures
\pysigline
{\sphinxbfcode{\sphinxupquote{xprobes}}\sphinxbfcode{\sphinxupquote{\DUrole{p}{:}\DUrole{w}{ }int\DUrole{w}{ }\DUrole{p}{|}\DUrole{w}{ }None}}\sphinxbfcode{\sphinxupquote{\DUrole{w}{ }\DUrole{p}{=}\DUrole{w}{ }None}}}
\pysigstopsignatures
\end{fulllineitems}

\index{yprobes (amiet\_self\_noise.io\_utils.ConfigData attribute)@\spxentry{yprobes}\spxextra{amiet\_self\_noise.io\_utils.ConfigData attribute}}

\begin{fulllineitems}
\phantomsection\label{\detokenize{io:amiet_self_noise.io_utils.ConfigData.yprobes}}
\pysigstartsignatures
\pysigline
{\sphinxbfcode{\sphinxupquote{yprobes}}\sphinxbfcode{\sphinxupquote{\DUrole{p}{:}\DUrole{w}{ }int\DUrole{w}{ }\DUrole{p}{|}\DUrole{w}{ }None}}\sphinxbfcode{\sphinxupquote{\DUrole{w}{ }\DUrole{p}{=}\DUrole{w}{ }None}}}
\pysigstopsignatures
\end{fulllineitems}


\end{fulllineitems}

\index{InputData (class in amiet\_self\_noise.io\_utils)@\spxentry{InputData}\spxextra{class in amiet\_self\_noise.io\_utils}}

\begin{fulllineitems}
\phantomsection\label{\detokenize{io:amiet_self_noise.io_utils.InputData}}
\pysigstartsignatures
\pysiglinewithargsret
{\sphinxbfcode{\sphinxupquote{\DUrole{k}{class}\DUrole{w}{ }}}\sphinxcode{\sphinxupquote{amiet\_self\_noise.io\_utils.}}\sphinxbfcode{\sphinxupquote{InputData}}}
{\sphinxparam{\DUrole{n}{config\_path}\DUrole{p}{:}\DUrole{w}{ }\DUrole{n}{str}}\sphinxparamcomma \sphinxparam{\DUrole{n}{normalize}\DUrole{p}{:}\DUrole{w}{ }\DUrole{n}{bool}\DUrole{w}{ }\DUrole{o}{=}\DUrole{w}{ }\DUrole{default_value}{True}}}
{}
\pysigstopsignatures
\sphinxAtStartPar
Bases: \sphinxcode{\sphinxupquote{object}}

\sphinxAtStartPar
Class to hold input data for the Amiet self\sphinxhyphen{}noise model.
This is the main entry point for the \sphinxcode{\sphinxupquote{amiet\_self\_noise}} package. This class reads
the configuration from a YAML file and loads the data based on the specified type.

\begin{sphinxadmonition}{note}{Note:}
\sphinxAtStartPar
The user should only really need to modify the configuration file to run all the
desired analyses. If this is not the case, please contact the developers.
\end{sphinxadmonition}

\begin{sphinxadmonition}{warning}{Warning:}
\sphinxAtStartPar
This class currently only supports DNS data. Other data types may be added in
the future.
\end{sphinxadmonition}
\begin{quote}\begin{description}
\sphinxlineitem{Parameters}\begin{itemize}
\item {} 
\sphinxAtStartPar
\sphinxstyleliteralstrong{\sphinxupquote{config\_path}} (\sphinxstyleliteralemphasis{\sphinxupquote{str}}) \textendash{} 
\sphinxAtStartPar
The path to the configuration file in YAML format. An example of the
configuration file is:

\begin{sphinxVerbatim}[commandchars=\\\{\}]
\PYG{n+nt}{L}\PYG{p}{:}\PYG{+w}{ }\PYG{l+lScalar+lScalarPlain}{10.0}
\PYG{n+nt}{T}\PYG{p}{:}\PYG{+w}{ }\PYG{l+lScalar+lScalarPlain}{300.0}
\PYG{n+nt}{U0}\PYG{p}{:}\PYG{+w}{ }\PYG{l+lScalar+lScalarPlain}{340.29}
\PYG{n+nt}{b}\PYG{p}{:}\PYG{+w}{ }\PYG{l+lScalar+lScalarPlain}{1.0}
\PYG{n+nt}{data\PYGZus{}path}\PYG{p}{:}\PYG{+w}{ }\PYG{l+lScalar+lScalarPlain}{/PATH/TO/DATA.h5}
\PYG{n+nt}{data\PYGZus{}type}\PYG{p}{:}\PYG{+w}{ }\PYG{l+lScalar+lScalarPlain}{dns}
\PYG{n+nt}{mesh\PYGZus{}path}\PYG{p}{:}\PYG{+w}{ }\PYG{l+lScalar+lScalarPlain}{/PATH/TO/MESH.h5}
\PYG{n+nt}{obs}\PYG{p}{:}
\PYG{p+pIndicator}{\PYGZhy{}}\PYG{+w}{ }\PYG{p+pIndicator}{[}\PYG{n+nv}{0.0}\PYG{p+pIndicator}{,}\PYG{+w}{ }\PYG{n+nv}{0.0}\PYG{p+pIndicator}{,}\PYG{+w}{ }\PYG{n+nv}{0.0}\PYG{p+pIndicator}{]}
\PYG{p+pIndicator}{\PYGZhy{}}\PYG{+w}{ }\PYG{p+pIndicator}{[}\PYG{n+nv}{1.0}\PYG{p+pIndicator}{,}\PYG{+w}{ }\PYG{n+nv}{1.0}\PYG{p+pIndicator}{,}\PYG{+w}{ }\PYG{n+nv}{0.0}\PYG{p+pIndicator}{]}
\PYG{n+nt}{xprobes}\PYG{p}{:}\PYG{+w}{ }\PYG{l+lScalar+lScalarPlain}{0}
\PYG{n+nt}{yprobes}\PYG{p}{:}\PYG{+w}{ }\PYG{l+lScalar+lScalarPlain}{null}\PYG{+w}{ }\PYG{c+c1}{\PYGZsh{} use all probes in y direction}
\end{sphinxVerbatim}

\sphinxAtStartPar
More information on the configuration file can be found in the {\hyperref[\detokenize{usage:target-to-input-files}]{\sphinxcrossref{\DUrole{std}{\DUrole{std-ref}{dedicated
page}}}}} in the documentation.


\item {} 
\sphinxAtStartPar
\sphinxstyleliteralstrong{\sphinxupquote{normalize}} (\sphinxstyleliteralemphasis{\sphinxupquote{bool}}\sphinxstyleliteralemphasis{\sphinxupquote{, }}\sphinxstyleliteralemphasis{\sphinxupquote{optional}}) \textendash{} If True, the data will be de\sphinxhyphen{}normalized using the configuration data.

\end{itemize}

\end{description}\end{quote}
\index{config (amiet\_self\_noise.io\_utils.InputData attribute)@\spxentry{config}\spxextra{amiet\_self\_noise.io\_utils.InputData attribute}}

\begin{fulllineitems}
\phantomsection\label{\detokenize{io:amiet_self_noise.io_utils.InputData.config}}
\pysigstartsignatures
\pysigline
{\sphinxbfcode{\sphinxupquote{config}}}
\pysigstopsignatures
\sphinxAtStartPar
The configuration data as a {\hyperref[\detokenize{io:amiet_self_noise.io_utils.ConfigData}]{\sphinxcrossref{\sphinxcode{\sphinxupquote{ConfigData}}}}} object.
\begin{quote}\begin{description}
\sphinxlineitem{Type}
\sphinxAtStartPar
{\hyperref[\detokenize{io:amiet_self_noise.io_utils.ConfigData}]{\sphinxcrossref{ConfigData}}}

\end{description}\end{quote}

\end{fulllineitems}

\index{pos (amiet\_self\_noise.io\_utils.InputData attribute)@\spxentry{pos}\spxextra{amiet\_self\_noise.io\_utils.InputData attribute}}

\begin{fulllineitems}
\phantomsection\label{\detokenize{io:amiet_self_noise.io_utils.InputData.pos}}
\pysigstartsignatures
\pysigline
{\sphinxbfcode{\sphinxupquote{pos}}}
\pysigstopsignatures
\sphinxAtStartPar
The positions of the sensors as a numpy array, shape (n\_sensors, 3).
\begin{quote}\begin{description}
\sphinxlineitem{Type}
\sphinxAtStartPar
np.array

\end{description}\end{quote}

\end{fulllineitems}

\index{pressure (amiet\_self\_noise.io\_utils.InputData attribute)@\spxentry{pressure}\spxextra{amiet\_self\_noise.io\_utils.InputData attribute}}

\begin{fulllineitems}
\phantomsection\label{\detokenize{io:amiet_self_noise.io_utils.InputData.pressure}}
\pysigstartsignatures
\pysigline
{\sphinxbfcode{\sphinxupquote{pressure}}}
\pysigstopsignatures
\sphinxAtStartPar
The pressure data as a numpy array, shape (n\_time\_steps, n\_sensors).
\begin{quote}\begin{description}
\sphinxlineitem{Type}
\sphinxAtStartPar
np.array

\end{description}\end{quote}

\end{fulllineitems}

\index{fs (amiet\_self\_noise.io\_utils.InputData attribute)@\spxentry{fs}\spxextra{amiet\_self\_noise.io\_utils.InputData attribute}}

\begin{fulllineitems}
\phantomsection\label{\detokenize{io:amiet_self_noise.io_utils.InputData.fs}}
\pysigstartsignatures
\pysigline
{\sphinxbfcode{\sphinxupquote{fs}}}
\pysigstopsignatures
\sphinxAtStartPar
The sampling frequency in Hz, derived from the data file.
\begin{quote}\begin{description}
\sphinxlineitem{Type}
\sphinxAtStartPar
float

\end{description}\end{quote}

\end{fulllineitems}

\index{print\_summary() (amiet\_self\_noise.io\_utils.InputData method)@\spxentry{print\_summary()}\spxextra{amiet\_self\_noise.io\_utils.InputData method}}

\begin{fulllineitems}
\phantomsection\label{\detokenize{io:amiet_self_noise.io_utils.InputData.print_summary}}
\pysigstartsignatures
\pysiglinewithargsret
{\sphinxbfcode{\sphinxupquote{print\_summary}}}
{\sphinxparam{\DUrole{n}{console}\DUrole{p}{:}\DUrole{w}{ }\DUrole{n}{Console}\DUrole{w}{ }\DUrole{o}{=}\DUrole{w}{ }\DUrole{default_value}{None}}}
{{ $\rightarrow$ None}}
\pysigstopsignatures
\sphinxAtStartPar
Print a detailed summary of the InputData configuration and loaded data.
\begin{quote}\begin{description}
\sphinxlineitem{Parameters}
\sphinxAtStartPar
\sphinxstyleliteralstrong{\sphinxupquote{console}} (\sphinxstyleliteralemphasis{\sphinxupquote{rich.console.Console}}\sphinxstyleliteralemphasis{\sphinxupquote{, }}\sphinxstyleliteralemphasis{\sphinxupquote{optional}}) \textendash{} Console instance to use for printing. If None, creates a new one.

\end{description}\end{quote}

\end{fulllineitems}


\end{fulllineitems}

\index{SensorData (class in amiet\_self\_noise.io\_utils)@\spxentry{SensorData}\spxextra{class in amiet\_self\_noise.io\_utils}}

\begin{fulllineitems}
\phantomsection\label{\detokenize{io:amiet_self_noise.io_utils.SensorData}}
\pysigstartsignatures
\pysiglinewithargsret
{\sphinxbfcode{\sphinxupquote{\DUrole{k}{class}\DUrole{w}{ }}}\sphinxcode{\sphinxupquote{amiet\_self\_noise.io\_utils.}}\sphinxbfcode{\sphinxupquote{SensorData}}}
{\sphinxparam{\DUrole{n}{pressure}\DUrole{p}{:}\DUrole{w}{ }\DUrole{n}{array}}\sphinxparamcomma \sphinxparam{\DUrole{n}{fs}\DUrole{p}{:}\DUrole{w}{ }\DUrole{n}{float}}\sphinxparamcomma \sphinxparam{\DUrole{n}{position}\DUrole{p}{:}\DUrole{w}{ }\DUrole{n}{array}\DUrole{w}{ }\DUrole{o}{=}\DUrole{w}{ }\DUrole{default_value}{None}}}
{}
\pysigstopsignatures
\sphinxAtStartPar
Bases: \sphinxcode{\sphinxupquote{object}}

\sphinxAtStartPar
Class to hold sensor data.
\begin{quote}\begin{description}
\sphinxlineitem{Parameters}\begin{itemize}
\item {} 
\sphinxAtStartPar
\sphinxstyleliteralstrong{\sphinxupquote{pressure}} (\sphinxstyleliteralemphasis{\sphinxupquote{np.array}}) \textendash{} Pressure data as a numpy array.

\item {} 
\sphinxAtStartPar
\sphinxstyleliteralstrong{\sphinxupquote{fs}} (\sphinxstyleliteralemphasis{\sphinxupquote{float}}) \textendash{} Sampling frequency in Hz.

\item {} 
\sphinxAtStartPar
\sphinxstyleliteralstrong{\sphinxupquote{position}} (\sphinxstyleliteralemphasis{\sphinxupquote{np.array}}\sphinxstyleliteralemphasis{\sphinxupquote{, }}\sphinxstyleliteralemphasis{\sphinxupquote{optional}}) \textendash{} Position as a numpy array. Default is None.

\end{itemize}

\end{description}\end{quote}
\index{fs (amiet\_self\_noise.io\_utils.SensorData attribute)@\spxentry{fs}\spxextra{amiet\_self\_noise.io\_utils.SensorData attribute}}

\begin{fulllineitems}
\phantomsection\label{\detokenize{io:amiet_self_noise.io_utils.SensorData.fs}}
\pysigstartsignatures
\pysigline
{\sphinxbfcode{\sphinxupquote{fs}}\sphinxbfcode{\sphinxupquote{\DUrole{p}{:}\DUrole{w}{ }float}}}
\pysigstopsignatures
\end{fulllineitems}

\index{position (amiet\_self\_noise.io\_utils.SensorData attribute)@\spxentry{position}\spxextra{amiet\_self\_noise.io\_utils.SensorData attribute}}

\begin{fulllineitems}
\phantomsection\label{\detokenize{io:amiet_self_noise.io_utils.SensorData.position}}
\pysigstartsignatures
\pysigline
{\sphinxbfcode{\sphinxupquote{position}}\sphinxbfcode{\sphinxupquote{\DUrole{p}{:}\DUrole{w}{ }array}}\sphinxbfcode{\sphinxupquote{\DUrole{w}{ }\DUrole{p}{=}\DUrole{w}{ }None}}}
\pysigstopsignatures
\end{fulllineitems}

\index{pressure (amiet\_self\_noise.io\_utils.SensorData attribute)@\spxentry{pressure}\spxextra{amiet\_self\_noise.io\_utils.SensorData attribute}}

\begin{fulllineitems}
\phantomsection\label{\detokenize{io:amiet_self_noise.io_utils.SensorData.pressure}}
\pysigstartsignatures
\pysigline
{\sphinxbfcode{\sphinxupquote{pressure}}\sphinxbfcode{\sphinxupquote{\DUrole{p}{:}\DUrole{w}{ }array}}}
\pysigstopsignatures
\end{fulllineitems}


\end{fulllineitems}

\index{read\_pressure\_data() (in module amiet\_self\_noise.io\_utils)@\spxentry{read\_pressure\_data()}\spxextra{in module amiet\_self\_noise.io\_utils}}

\begin{fulllineitems}
\phantomsection\label{\detokenize{io:amiet_self_noise.io_utils.read_pressure_data}}
\pysigstartsignatures
\pysiglinewithargsret
{\sphinxcode{\sphinxupquote{amiet\_self\_noise.io\_utils.}}\sphinxbfcode{\sphinxupquote{read\_pressure\_data}}}
{\sphinxparam{\DUrole{n}{path}\DUrole{p}{:}\DUrole{w}{ }\DUrole{n}{str}}\sphinxparamcomma \sphinxparam{\DUrole{n}{pressure\_key}\DUrole{p}{:}\DUrole{w}{ }\DUrole{n}{str}\DUrole{w}{ }\DUrole{o}{=}\DUrole{w}{ }\DUrole{default_value}{\textquotesingle{}pressure\textquotesingle{}}}\sphinxparamcomma \sphinxparam{\DUrole{n}{time\_key}\DUrole{p}{:}\DUrole{w}{ }\DUrole{n}{str}\DUrole{w}{ }\DUrole{o}{=}\DUrole{w}{ }\DUrole{default_value}{\textquotesingle{}time\textquotesingle{}}}}
{{ $\rightarrow$ Tuple\DUrole{p}{{[}}array\DUrole{p}{,}\DUrole{w}{ }array\DUrole{p}{{]}}}}
\pysigstopsignatures
\sphinxAtStartPar
Read pressure data from a .h5 file and return the data as a numpy array.
\begin{quote}\begin{description}
\sphinxlineitem{Parameters}\begin{itemize}
\item {} 
\sphinxAtStartPar
\sphinxstyleliteralstrong{\sphinxupquote{path}} (\sphinxstyleliteralemphasis{\sphinxupquote{str}}) \textendash{} The path to the .h5 file.

\item {} 
\sphinxAtStartPar
\sphinxstyleliteralstrong{\sphinxupquote{pressure\_key}} (\sphinxstyleliteralemphasis{\sphinxupquote{str}}) \textendash{} The key for the pressure data in the .h5 file. Default is ‘pressure’.

\item {} 
\sphinxAtStartPar
\sphinxstyleliteralstrong{\sphinxupquote{time\_key}} (\sphinxstyleliteralemphasis{\sphinxupquote{str}}) \textendash{} The key for the time data in the .h5 file. Default is ‘time’.

\end{itemize}

\sphinxlineitem{Returns}
\sphinxAtStartPar
\begin{itemize}
\item {} 
\sphinxAtStartPar
\sphinxstylestrong{pressure} (\sphinxstyleemphasis{np.array}) \textendash{} The pressure data as a numpy array.

\item {} 
\sphinxAtStartPar
\sphinxstylestrong{time} (\sphinxstyleemphasis{np.array}) \textendash{} The time data as a numpy array.

\end{itemize}


\end{description}\end{quote}

\end{fulllineitems}


\sphinxstepscope


\section{radiation\_integral module}
\label{\detokenize{radiation_integral:module-amiet_self_noise.radiation_integral}}\label{\detokenize{radiation_integral:radiation-integral-module}}\label{\detokenize{radiation_integral::doc}}\index{module@\spxentry{module}!amiet\_self\_noise.radiation\_integral@\spxentry{amiet\_self\_noise.radiation\_integral}}\index{amiet\_self\_noise.radiation\_integral@\spxentry{amiet\_self\_noise.radiation\_integral}!module@\spxentry{module}}\index{compute\_radiation\_integral() (in module amiet\_self\_noise.radiation\_integral)@\spxentry{compute\_radiation\_integral()}\spxextra{in module amiet\_self\_noise.radiation\_integral}}

\begin{fulllineitems}
\phantomsection\label{\detokenize{radiation_integral:amiet_self_noise.radiation_integral.compute_radiation_integral}}
\pysigstartsignatures
\pysiglinewithargsret
{\sphinxcode{\sphinxupquote{amiet\_self\_noise.radiation\_integral.}}\sphinxbfcode{\sphinxupquote{compute\_radiation\_integral}}}
{\sphinxparam{\DUrole{n}{omega\_array}}\sphinxparamcomma \sphinxparam{\DUrole{n}{U0}}\sphinxparamcomma \sphinxparam{\DUrole{n}{c0}}\sphinxparamcomma \sphinxparam{\DUrole{n}{x1}}\sphinxparamcomma \sphinxparam{\DUrole{n}{S0}}\sphinxparamcomma \sphinxparam{\DUrole{n}{M0}}\sphinxparamcomma \sphinxparam{\DUrole{n}{b}}\sphinxparamcomma \sphinxparam{\DUrole{n}{alpha}\DUrole{o}{=}\DUrole{default_value}{1.0}}\sphinxparamcomma \sphinxparam{\DUrole{n}{a\_param}\DUrole{o}{=}\DUrole{default_value}{None}}}
{}
\pysigstopsignatures
\sphinxAtStartPar
Compute the Amiet radiation integral for airfoil trailing edge noise.

\sphinxAtStartPar
This function calculates the complex\sphinxhyphen{}valued radiation integral that
represents the acoustic transfer function from surface pressure
fluctuations to far\sphinxhyphen{}field sound pressure in the Amiet model. The
integral accounts for the acoustic scattering effects at the airfoil
trailing edge.
\begin{quote}\begin{description}
\sphinxlineitem{Parameters}\begin{itemize}
\item {} 
\sphinxAtStartPar
\sphinxstyleliteralstrong{\sphinxupquote{omega\_array}} (\sphinxstyleliteralemphasis{\sphinxupquote{array\_like}}) \textendash{} Angular frequency array in rad/s, shape (n\_freq,).

\item {} 
\sphinxAtStartPar
\sphinxstyleliteralstrong{\sphinxupquote{U0}} (\sphinxstyleliteralemphasis{\sphinxupquote{float}}) \textendash{} Free\sphinxhyphen{}stream velocity in m/s.

\item {} 
\sphinxAtStartPar
\sphinxstyleliteralstrong{\sphinxupquote{c0}} (\sphinxstyleliteralemphasis{\sphinxupquote{float}}) \textendash{} Speed of sound in m/s.

\item {} 
\sphinxAtStartPar
\sphinxstyleliteralstrong{\sphinxupquote{x1}} (\sphinxstyleliteralemphasis{\sphinxupquote{float}}) \textendash{} Observer x\sphinxhyphen{}coordinate (streamwise direction) in m.

\item {} 
\sphinxAtStartPar
\sphinxstyleliteralstrong{\sphinxupquote{S0}} (\sphinxstyleliteralemphasis{\sphinxupquote{float}}) \textendash{} Observer distance from trailing edge in m.

\item {} 
\sphinxAtStartPar
\sphinxstyleliteralstrong{\sphinxupquote{M0}} (\sphinxstyleliteralemphasis{\sphinxupquote{float}}) \textendash{} Free\sphinxhyphen{}stream Mach number, dimensionless.

\item {} 
\sphinxAtStartPar
\sphinxstyleliteralstrong{\sphinxupquote{b}} (\sphinxstyleliteralemphasis{\sphinxupquote{float}}) \textendash{} Airfoil semi\sphinxhyphen{}chord (half chord length) in m.

\item {} 
\sphinxAtStartPar
\sphinxstyleliteralstrong{\sphinxupquote{alpha}} (\sphinxstyleliteralemphasis{\sphinxupquote{float}}\sphinxstyleliteralemphasis{\sphinxupquote{, }}\sphinxstyleliteralemphasis{\sphinxupquote{optional}}) \textendash{} Convection velocity ratio Uc/U0, where Uc is the convection
velocity of turbulent eddies. Default is 1.0.

\item {} 
\sphinxAtStartPar
\sphinxstyleliteralstrong{\sphinxupquote{a\_param}} (\sphinxstyleliteralemphasis{\sphinxupquote{float}}\sphinxstyleliteralemphasis{\sphinxupquote{, }}\sphinxstyleliteralemphasis{\sphinxupquote{optional}}) \textendash{} Alternative parameter for convection velocity ratio. If provided,
overrides the alpha parameter. Default is None.

\end{itemize}

\sphinxlineitem{Returns}
\sphinxAtStartPar
\sphinxstylestrong{I} \textendash{} Complex radiation integral values, shape (n\_freq,).
The magnitude squared \(\vert I\vert^2\) represents the acoustic efficiency
of the trailing edge scattering process.

\sphinxlineitem{Return type}
\sphinxAtStartPar
ndarray, complex

\end{description}\end{quote}

\begin{sphinxadmonition}{note}{Note:}
\sphinxAtStartPar
The radiation integral is computed as \(I = L_1 + L_2\), where:
\begin{itemize}
\item {} 
\sphinxAtStartPar
\(L_1\) represents the leading edge contribution to the scattering

\item {} 
\sphinxAtStartPar
\(L_2\) represents the trailing edge contribution to the scattering

\end{itemize}

\sphinxAtStartPar
The computation involves:
\begin{enumerate}
\sphinxsetlistlabels{\arabic}{enumi}{enumii}{}{.}%
\item {} 
\sphinxAtStartPar
Calculation of dimensionless parameters:

\end{enumerate}
\begin{itemize}
\item {} 
\sphinxAtStartPar
\(\mu = kb/\beta^2\) (reduced frequency)

\item {} 
\sphinxAtStartPar
\(\beta^2 = 1 - M_0^2\) (compressibility factor)

\item {} 
\sphinxAtStartPar
\(k = \omega/c_0\) (acoustic wavenumber)

\end{itemize}
\begin{enumerate}
\sphinxsetlistlabels{\arabic}{enumi}{enumii}{}{.}%
\setcounter{enumi}{1}
\item {} 
\sphinxAtStartPar
Evaluation of Fresnel integrals through the \(E^\star\) function

\end{enumerate}

\sphinxAtStartPar
The implementation handles numerical singularities by applying
small regularization values (1e\sphinxhyphen{}12) when parameters approach zero.
\end{sphinxadmonition}

\begin{sphinxadmonition}{warning}{Warning:}
\sphinxAtStartPar
For \(\omega = 0\), the function returns 0 as a placeholder. The correct
asymptotic behavior at low frequencies is not yet implemented.
\end{sphinxadmonition}
\subsubsection*{Examples}

\begin{sphinxVerbatim}[commandchars=\\\{\}]
\PYG{n}{omega} \PYG{o}{=} \PYG{n}{np}\PYG{o}{.}\PYG{n}{array}\PYG{p}{(}\PYG{p}{[}\PYG{l+m+mi}{100}\PYG{p}{,} \PYG{l+m+mi}{1000}\PYG{p}{,} \PYG{l+m+mi}{10000}\PYG{p}{]}\PYG{p}{)}  \PYG{c+c1}{\PYGZsh{} Angular frequencies}
\PYG{n}{I} \PYG{o}{=} \PYG{n}{compute\PYGZus{}radiation\PYGZus{}integral}\PYG{p}{(}\PYG{n}{omega}\PYG{p}{,} \PYG{n}{U0}\PYG{o}{=}\PYG{l+m+mi}{50}\PYG{p}{,} \PYG{n}{c0}\PYG{o}{=}\PYG{l+m+mi}{343}\PYG{p}{,} \PYG{n}{x1}\PYG{o}{=}\PYG{l+m+mf}{1.0}\PYG{p}{,}
                            \PYG{n}{S0}\PYG{o}{=}\PYG{l+m+mf}{10.0}\PYG{p}{,} \PYG{n}{M0}\PYG{o}{=}\PYG{l+m+mf}{0.15}\PYG{p}{,} \PYG{n}{b}\PYG{o}{=}\PYG{l+m+mf}{0.1}\PYG{p}{)}
\PYG{n}{efficiency} \PYG{o}{=} \PYG{n}{np}\PYG{o}{.}\PYG{n}{abs}\PYG{p}{(}\PYG{n}{I}\PYG{p}{)}\PYG{o}{*}\PYG{o}{*}\PYG{l+m+mi}{2}  \PYG{c+c1}{\PYGZsh{} Acoustic efficiency}
\end{sphinxVerbatim}
\subsubsection*{References}

\end{fulllineitems}



\chapter{Usage example}
\label{\detokenize{index:usage-example}}
\sphinxAtStartPar
Using \sphinxcode{\sphinxupquote{amiet\_self\_noise}} in your projects is as easy as writing four lines of code:

\begin{sphinxVerbatim}[commandchars=\\\{\}]
\PYG{c+c1}{\PYGZsh{} 1. import the module}
\PYG{k+kn}{import}\PYG{+w}{ }\PYG{n+nn}{amiet\PYGZus{}self\PYGZus{}noise}\PYG{+w}{ }\PYG{k}{as}\PYG{+w}{ }\PYG{n+nn}{asn}
\PYG{c+c1}{\PYGZsh{} 2. read the data from the configuration file}
\PYG{n}{input\PYGZus{}data} \PYG{o}{=} \PYG{n}{asn}\PYG{o}{.}\PYG{n}{io\PYGZus{}utils}\PYG{o}{.}\PYG{n}{InputData}\PYG{p}{(}\PYG{l+s+s2}{\PYGZdq{}}\PYG{l+s+s2}{config.yaml}\PYG{l+s+s2}{\PYGZdq{}}\PYG{p}{,} \PYG{n}{normalize}\PYG{o}{=}\PYG{k+kc}{True}\PYG{p}{)}
\PYG{c+c1}{\PYGZsh{} 3. initialize the model}
\PYG{n}{model} \PYG{o}{=} \PYG{n}{asn}\PYG{o}{.}\PYG{n}{amiet\PYGZus{}model}\PYG{o}{.}\PYG{n}{AmietModel}\PYG{p}{(}\PYG{n}{input\PYGZus{}data}\PYG{p}{)}
\PYG{c+c1}{\PYGZsh{} 4. compute the PSD}
\PYG{n}{f}\PYG{p}{,} \PYG{n}{psd} \PYG{o}{=} \PYG{n}{model}\PYG{o}{.}\PYG{n}{compute\PYGZus{}psd}\PYG{p}{(}\PYG{p}{)}
\end{sphinxVerbatim}

\sphinxAtStartPar
Read more about the configuration files in the {\hyperref[\detokenize{usage:target-to-input-files}]{\sphinxcrossref{\DUrole{std}{\DUrole{std-ref}{dedicated page}}}}}.


\renewcommand{\indexname}{Python Module Index}
\begin{sphinxtheindex}
\let\bigletter\sphinxstyleindexlettergroup
\bigletter{a}
\item\relax\sphinxstyleindexentry{amiet\_self\_noise.amiet\_model}\sphinxstyleindexpageref{amiet_model:\detokenize{module-amiet_self_noise.amiet_model}}
\item\relax\sphinxstyleindexentry{amiet\_self\_noise.io\_utils}\sphinxstyleindexpageref{io:\detokenize{module-amiet_self_noise.io_utils}}
\item\relax\sphinxstyleindexentry{amiet\_self\_noise.preproc}\sphinxstyleindexpageref{preproc:\detokenize{module-amiet_self_noise.preproc}}
\item\relax\sphinxstyleindexentry{amiet\_self\_noise.radiation\_integral}\sphinxstyleindexpageref{radiation_integral:\detokenize{module-amiet_self_noise.radiation_integral}}
\end{sphinxtheindex}

\renewcommand{\indexname}{Index}
\printindex
\end{document}